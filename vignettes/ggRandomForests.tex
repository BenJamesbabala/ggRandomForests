\documentclass[nojss]{jss}

%%%%%%%%%%%%%%%%%%%%%%%%%%%%%%
%% declarations for jss.cls %%%%%%%%%%%%%%%%%%%%%%%%%%%%%%%%%%%%%%%%%%
%%%%%%%%%%%%%%%%%%%%%%%%%%%%%%
%\VignetteEngine{knitr::knitr}
%\VignetteIndexEntry{ggRandomForests}
%\VignetteIndexEntry{A guide to Random Forests with the randomForestSRC and ggRandomForests packages}                                          
%\VignetteKeywords{random forest, survival, classification, regression, VIMP, minimal depth}                                                                   
%\VignetteDepends{ggRandomForests, randomForestsSRC, xtable, survival}                                      
%\VignettePackage{ggRandomForests} 

%% almost as usual
\author{John Ehrlinger\\Cleveland Clinic} %\And \\Plus Affiliation}
\title{{\pkg{ggRandomForests}}: Visually Exploring Random Forests in \proglang{R}}

%% for pretty printing and a nice hypersummary also set:
\Plainauthor{John Ehrlinger} %% comma-separated
\Plaintitle{ggRandomForests: Visually Exploring Random Forests} %% without formatting
\Shorttitle{{\pkg{ggRandomForests}}: Visually Exploring Random Forests in \proglang{R}}

%% an abstract and keywords
\Abstract{ 
We introduce the \proglang{R} package \pkg{ggRandomForests}.}
\Keywords{random forest, survival, classification, regression, VIMP, minimal depth}
\Plainkeywords{random forest, survival, classification, regression, VIMP, minimal depth}
%% at least one keyword must be supplied

%% publication information
%% NOTE: Typically, this can be left commented and will be filled out by the technical editor
%% \Volume{13}
%% \Issue{9}
%% \Month{September}
%% \Year{2004}
%% \Submitdate{2004-09-29}
%% \Acceptdate{2004-09-29}

%% The address of (at least) one author should be given
%% in the following format:
\Address{
  John Ehrlinger\\
  Quantitative Health Sciences\\
  Lerner Research Institute\\
  Cleveland Clinic\\
  9500 Euclid Ave\\
  Cleveland, Ohio 44195\\
%  Telephone: +41/0/44634-4643 \\
%  Fax: +41/0/44634-4386 \\
  E-mail: \email{john.ehrlinger@gmail.com}\\
  URL: \url{http://www.biostat.uzh.ch/aboutus/people/rufibach.html}
}

%% It is also possible to add a telephone and fax number
%% before the e-mail in the following format:
%% Telephone: +43/1/31336-5053
%% Fax: +43/1/31336-734

%% for those who use Sweave please include the following line (with % symbols):
%% need no \usepackage{Sweave.sty}

%% end of declarations %%%%%%%%%%%%%%%%%%%%%%%%%%%%%%%%%%%%%%%%%%%%%%%

\usepackage{longtable}
\usepackage{lscape}

% keep code formatting
%\SweaveOpts{keep.source = TRUE}



\begin{document}

%% include your article here, just as usual
%% Note that you should use the \pkg{}, \proglang{} and \code{} commands.



% -----------------------------------------------------
\section{About this document}
% -----------------------------------------------------
This document is an introduction to the \proglang{R} package
\pkg{reporttools} \citep{reporttools} based on \citet{Rufibach:2009}. It aims to provide a detailed user guide 
based on simple, reproducible worked examples. The \pkg{reporttools} package is available from the
Comprehensive \proglang{R} Archive Network via \url{http://CRAN.R-project.org/package=reporttools}.

Compared to the package that was the base of \citet{Rufibach:2009} some minor modifications and additions 
have been made to the package. All changes were intended to be backwards compatible.


% -----------------------------------------------------
\section{Introduction}
% -----------------------------------------------------
In statistical analysis reports and medical publications it is common practice to start statistical analyses with 
displays of descriptive statistics of patient characteristics and further important variables. 
The purpose of these tables is (1) to get an idea about basic features of the data and (2) especially in 
analysis reports, data checking.
Tables of descriptive statistics may take different formats, depending on the type of 
variables to be displayed, which descriptive statistics are to be reported, and whether statistics should be given
for all observations jointly or separately for the levels of a given factor, such as e.g. treatment arm. 
To be able to efficiently generate these recurring parts of analyses when combining {\LaTeX}
\citep{knuth_84, lamport_94} with \proglang{R} \citep{R} code via \code{Sweave} \citep{leisch_02}, the \proglang{R} package \pkg{reporttools} 
provides functions to generate {\LaTeX} tables of descriptive statistics for nominal, date,
and continuous variables.
The tables are set up as data frames in \proglang{R}, then translated into {\LaTeX} code
using the standard \proglang{R} package \pkg{xtable} \citep{xtable}. Using \code{Sweave}, these tables can be directly generated
in {\LaTeX} documents by invoking basically only one line of \proglang{R} code.

A package that offers somewhat related functionality is \pkg{r2lUniv} \citep{r2lUniv}.
In that package, each variable gets a separate section providing descriptive statistics and corresponding plots, 
depending on the type of variable that is analyzed. The default functions in \pkg{r2lUniv}
directly generate an entire {\LaTeX} document, thereby reporting the descriptive statistics of only about three variables
on one single page. Set up this way, it seems difficult to efficiently merge plain text and data analysis using \pkg{r2lUniv}.
Additionally, we are not aware of a possibility in \pkg{r2lUniv} to compare a given variable between different groups in an
easy way. The functions introduced in \pkg{reporttools} aim at closing these gaps. The tabulating functions 
in \pkg{reporttools} can be applied inside a \code{.Rnw} document combining {\LaTeX} plain text and data 
analyses in \proglang{R}.

A common feature in data analysis is, that one needs to provide descriptive statistics of a large set of variables thereby
generating tables that are larger than one page. The functions described here have by default 
the \code{tabular.environment}-option
in the implicitly used \proglang{R} function \code{print.xtable} set to \code{longtable}. Together
with the {\LaTeX} package \pkg{longtable}, setting this option generates tables that may range over more than a 
page, without additional specification or programming effort.

The package \pkg{reporttools} contains several additional functions useful in setting up analyses and 
writing reports. However, the emphasis in this article is on the tabulating functions. For a more detailed description
of further elements of the package we refer to its \proglang{R} help files.

In Section~\ref{sec: data} we introduce the dataset that is used to illustrate the tabulating functions 
in Section~\ref{sec: heart}. Some conclusions are drawn in Section~\ref{sec: concl}.

% -----------------------------------------------------
\section{Stanford heart transplantation data} \label{sec: data}
% -----------------------------------------------------
We illustrate our new functions using the Stanford heart transplantation dataset \code{jasa} in the 
standard \proglang{R} package \pkg{survival} \citep{survival}. 
This dataset provides some patient characteristics and the survival of patients on the waiting list for the 
Stanford heart transplantation program, see \cite{crowley_77} and the corresponding \proglang{R} help file for details.


% -----------------------------------------------------
\section{Descriptive statistics for heart transplantation data} \label{sec: heart}
% -----------------------------------------------------
To demonstrate the entire flexibility of \pkg{reporttools} we use the package to describe the Stanford heart
transplantation data.

\subsection{Nominal variables (factors)}
The following code lines invoke \pkg{reporttools} and prepare the variables from \code{jasa} for later use. 

























